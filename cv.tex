\documentclass[10pt]{article}
\usepackage{array, xcolor, lipsum, bibentry}
\usepackage[margin=2cm]{geometry}
\usepackage{hyperref}

\title{\bfseries\Huge \color{gray}{Prabal Agarwal}}
\author{
\begin{normalsize}pagarwal@mpi-inf.mpg.de\hspace{170pt}+49 151 4574 1373\end{normalsize}\\
\begin{normalsize}\texttt{mpi-inf.mpg.de/\textasciitilde pagarwal}\hspace{75pt}Room. 115A, MPII, Saarbr\"ucken, DE\end{normalsize}
}
\date{}
\definecolor{lightgray}{gray}{0.8}
\newcolumntype{L}{>{\raggedleft}p{0.14\textwidth}}
\newcolumntype{R}{p{0.8\textwidth}}
\newcommand\VRule{\color{lightgray}\vrule width 0.5pt}

\begin{document}
\maketitle

{\vspace{20pt}
\section*{Research Statement}
\begin{tabular}{L!{\VRule}R}
&

I am a masters student at International Max Plank Research School for Computer Science.
\end{tabular}


\section*{Education}
\begin{tabular}{L!{\VRule}R}
2015--Present&{\bf Pursuing Masters in Computer Science, IMPRS-CS, Germany}\vspace{5pt}\\
			& GPA 1.7 (ECTS grading scale)\\
			& {\bf Graduate Coursework:}\\
&{\it Information Retrieval and Data Mining, Advanced Topics in Information Retrieval,}\\
& {\it Database Systems, Elements of Statistical Learning, Social Media Analysis,}\\
& {\it Artificial Intelligence, Statistical Natural Language Processing}\\
\\
2011--2015&{\bf B.Tech. in Electrical Engineering, IIT-(BHU), Varanasi, India}\\ & GPA 8.23/10.0
\end{tabular}

\section*{Publications}
\begin{tabular}{L!{\VRule}R}
2017& Agarwal, Str\"otgen: \href{https://people.mpi-inf.mpg.de/~jstroetge/papers/2017-WWW-AgarwalStroetgen-Tiwiki.pdf}{\textit{Tiwiki: Searching Wikipedia with Temporal Constraints.}} TempWeb'17, Perth, Australia, April 3, 2017.\\
\end{tabular}
{\vspace{20pt}

\section*{Work Experience}
\begin{tabular}{L!{\VRule}R}
Currently&{\bf Research Assistant, Max Plank Institute for Informatics, Saarbr\"ucken, Germany}\\
	& In Databases and Information Systems Department led by Prof. Gerhard Weikum.\\[0.1cm]
& Explore and develop applications in the field of temporal information retrieval through \href{https://www.mpi-inf.mpg.de/departments/databases-and-information-systems/research/yago-naga/timesea/}{TimeSEA}.\\\\
2015 & {\bf Intern, Minewhat Inc.}\\
  & As a data science intern, I developed  product and category sales prediction models for e-commerce based clients. Subsequently, I developed a checkout prediction model based on customer behaviour and an anomaly detection model to identify interesting peaks in the sales of a product.
  \\\\
2014 & {\bf Intern, Wipro Technologies, Bangalore, India}\\
& In WiproSight team led by Mr. Akbar Ladak in the Chief Technology Office\\[0.1cm]
& Contributed to the development of the next generation e-commerce system by building  a facial emotion recognition system using a live video feed.
\\\\
2013 &\textbf{Research Intern, IIIT Hyderabad, India}\\
& in LTRC group led by Prof. Deepti Misra Sharma \\[0.1cm]
 & Worked on Hindi language based NLP tools to resolve modifier-antecedent  and entity type anaphora-antecedent relations.
 \\\\
\end{tabular}

\section*{Research Projects}
\begin{tabular}{L!{\VRule}R}
	Currently& {\bf Time in Named Entity Disambiguation}\\
	& Guided by Dr. Jannik Str\"otgen, Dr. Johannes Hoffart, and Prof. Gerhard Weikum \\[0.1cm]
	& As a part of Master's thesis, currently working on improving the state of the art named entity disambiguation system, \href{https://www.mpi-inf.mpg.de/departments/databases-and-information-systems/research/yago-naga/aida/}{AIDA}, leveraging temporal context. Additionally, working on to improve the overall system by computing  textual context similarity using word embeddings. \\ \\
	2016 & \href{http://tiwiki.mpi-inf.mpg.de}{\textbf{TIWIKI}}\\
	& In Databases and Information Systems Department led by Prof. Gerhard Weikum \\[0.1cm]
	& Developed a search engine for retrieving Wikipedia pages based on a query that can be augmented with temporal constraints. For a demo, click on the title\\\\
	2015 & {\bf Anaphora Resolution in the Context of Named Entities}\\
	& Guided by Prof. Rajeev Sangal, Director, IIT-(BHU), India\\[0.1cm]
	& As a part of Bachelor’s thesis, developed an event type anaphora-antecedent resolution system. Further combined this work with the existing entity type anaphora resolution system and improved the overall performance by adding semantic features. Additionally, studied the importance of context in the task of anaphora resolution.\\\\
	2014 & {\bf Vocaber}\\
	& Developed an application to recommend a sentiment score ordered list of synonyms for selected words in a text.\\\\
\end{tabular}

{\vspace{20pt}
\section*{Skills}
\begin{tabular}{L!{\VRule}R}
& \textit{Languages and Tools:} Java, Python, R, ElasticSearch, MongoDB, Apache Spark, Cassandra\\
& \textit{Operating Systems:} Linux, Windows\\
\end{tabular}

{\vspace{20pt}
\section*{Leadership and Miscellanea}
\begin{tabular}{L!{\VRule}R}
& Chairman, IEEE IIT-(BHU) for session 2013-14\\
& Chief Event Coordinator of Prastuti'15\\
& Won IEEE Darrel Chong Student Activity Award, 2014 (Gold position). 
\end{tabular}

\end{document}